\documentclass[9pt,twocolumn,twoside]{pnas-report}

\templatetype{pnasresearcharticle}

\usepackage{lipsum}

% set figures directory to be ./figures
\graphicspath{{./figures/}}

\title{Draft}

\author[a]{Žiga Trček}
\author[a]{Matej Urbas}
\author[a]{Jan Vasiljević}

\affil[a]{University of Ljubljana, Faculty of Computer and Information Science, Ve\v{c}na pot 113, SI-1000 Ljubljana, Slovenia}

\leadauthor{Jan Vasiljević}

\authordeclaration{All authors contributed equally to this work.}
\correspondingauthor{\textsuperscript{1}To whom correspondence should be addressed. E-mail: fine.author@email.com.}

\begin{abstract}
We study a phenomenon that we claim is important in a subject in which we claim many people are interested. We also claim that such ideas have been studied heavily both in network science and in other disciplines, although all prior work on this topic is horrible (though we will try to phrase that statement as politely as possible). One particular idea, which has some intriguing features but which either rarely has been studied or has only been studied in a crappy way before, is the one that we investigate in this project. Invoking minimal sarcasm, we study an extended version of this idea with our new approach, which we will claim to give universal results if we can get away with it. Our new approach has some bells and whistles that we study in our project, and we use computations, theory, and experiments to give important insights into a phenomenon that many people care about. We also give some gratuitous caveats so that peers will take us seriously, and we pray to our favorite deity that this approach is not equivalent to one that already exists (we forgot to check, but the authors need to graduate). We hope that our study, in addition to its intrinsic quality, will inspire future investigations, citations, and successful grants (and --- who knows? --- maybe even a Turing Award). In case you did not see it last time, we claim once again that our approach is universal.
\end{abstract}

\dates{The manuscript was compiled on \today}
\doi{\href{https://ucilnica.fri.uni-lj.si/course/view.php?id=183}{Introduction to Network Analysis} 2023/24}

\begin{document}

\maketitle
\thispagestyle{firststyle}
\ifthenelse{\boolean{shortarticle}}{\ifthenelse{\boolean{singlecolumn}}{\abscontentformatted}{\abscontent}}{}

\dropcap{P}{\bf roblem definition, motivation, background, contributions etc.}
\lipsum[1-4]

\begin{figure}[t]\centering%
	\includegraphics[width=0.9\linewidth]{examples}
	\includegraphics[width=\linewidth]{growth}
	\caption{Mandatory informative illustration highlighting main contributions.~\cite{Sub18a}}
	\label{fig:example}
\end{figure}

\section*{Related work}

Robustness of dependency networks is a topic that has been examined in several studies.
\cite{hafner2021robustness} explored the robustness of the npm package dependency network and highlighted its vulnerability to targeted attacks by identifying crucial nodes.
Attacks on such nodes could gravely affect a large portion of the network, as they are responsible for a significant number of dependencies.
Their findings suggest that while the network has crucial nodes, it is not highly vulnerable as they are mostly large and well-maintained projects.
Moreover, the network is trending towards a more robust state, as the number of dependencies is decreasing over time.

\cite{decan2018evolution} published a comparative analysis of dependency networks across 7 package ecosystems, including Cargo, CPAN, CRAN, npm, NuGet, Packagist, and RubyGems.
They proposed metrics to capture growth, changeability, reusability, and fragility.
They revealed that while the ecosystems are growing, a minority of packages are responsible for most updates and dependencies.
Furthermore, it is suggested that even transitive dependencies on unmaintained or obsolete packages can have a detrimental effect on the security and maintainability of the ecosystem.

\cite{tsakpinis2024accessibility} carried out a study on the accessibility of GitHub repositories for npm and PyPI libraries to understand the level of maintenance for them.
Their research showed that a significant portion of libraries lacked valid repository URLs, which hinders the ability to monitor vulnerabilities and maintain codebases.
An emphasis on the importance of maintaining valid repository URLs was made, as it is crucial for the sustainability of the open-source ecosystem.

Community detection has been proposed as a security analysis tool.
The found communities can provide great insight into the structure of the network and the relationships between packages, which helps assess the robustness of the network.
Understanding what packages are closely related and how they are connected can help identify potential vulnerabilities and dependencies that could be exploited by attackers \cite{hafner2021robustness, tsakpinis2024accessibility}.

\cite{korkmazrpackages} examine the dependency graph of R packages and the relationship between various metrics.
They find that centrality measures have a high correlation with the number of downloads and citations of a package.
Furthermore, package attributes such as number of authors and commits also have a positive impact on the number of downloads and citations.

Security vulnerability data analysis has been a large area of research in the recent years.
\cite{decan2018vulnerabilities} studied around 400 security reports from the npm network to understand how they are discovered and fixed, and how they affect the network.
They found that it takes around a year for a discovered vulnerability to be published publically, while it takes around 2 years after the discovery for the vulnerability to be fixed.
\cite{shahzad2012} study vulnerability life cycles on a large software vulnerability dataset, getting similar results.

\cite{HANIF2021103009} propose and evaluate novel machine learning approaches to vulnerability prediction in open-source software.
Some simple vulnerabilities are easily detected by their approach, but even more complex vulnerabilities can be detected with high accuracy.

\cite{hejderup2018} use call graphs instead of dependency graphs to detect dependencies in software.
They find that call graphs are efficient in aiding preliminary evaluation of security issues and their impact to other applications.
Furthermore, \cite{hejderup2022prazi} find that Cargo packages call only 40\% of their dependencies, which can lead to many unneeded security vulnerabilities.

\cite{ruohonen2021} examine around 200000 PyPI packages using static analysis and find that around 46\% of the packages have at least one security vulnerability.


\nocite{Kle00,Bou05,EB07,New08,For10,New12,FH16,PLC17,PDL18,Pei20}

\section*{Results}

{\bf Main results supported by math, plots, tables, diagrams etc.}
\lipsum[1]

\begin{table}[h]\centering%
	\caption{Table describing data or methods.}
	\begin{tabular}{lccccc}\toprule
	    & $n$ & $m$ & $\langle k\rangle$ & $\langle C\rangle$ & $\langle d\rangle$ \\\midrule
	    Fine network & $438\,920$ & $9\,742\,733$ & $44.4$ & $0.37$ & $6.19$ \\
	    Random graph & $438\,920$ & $9\,781\,609$ & $44.6$ & $0.00$ & $4.92$ \\\bottomrule
	\end{tabular}
	\label{tbl:example}
\end{table}

\lipsum[2-3]

\begin{figure}[t]\centering%
	\includegraphics[width=0.49\linewidth]{example1}
	\includegraphics[width=0.49\linewidth]{example2}
	\caption{Figure showing interesting examples.~\cite{Sub18a}}
\end{figure}

\lipsum[4-6]

\begin{figure}[b]\centering%
	\includegraphics[width=\linewidth]{distributions}
	\caption{Figure showing relevant results.~\cite{Sub18a}}
\end{figure}

\begin{figure}[t]\centering%
	\includegraphics[width=0.45\linewidth]{results1}\hskip12pt
	\includegraphics[width=0.45\linewidth]{results2}
	\caption{Another figure with results.~\cite{Sub18a}}
	\label{fig:example}
\end{figure}

\section*{Discussion}

{\bf Summary of results, main contributions, final conclusions, future work etc.}
\lipsum[1-3]

{\small

\section*{Methods}

{\bf Data, methods, algorithms etc.}
\lipsum[1]

\begin{equation}
	\phi_v = \Pr(X_{st}(v) = 1) = \Pr(X_{sv} = 1)\Pr(X_{vt} = 1)
	\label{eq:example}
\end{equation}

\lipsum[2]

\begin{algorithm}[H]
	\begin{algorithmic}[1]
		\Require graph $G$, cutoff $k_{min}$
		\Ensure power-law $\gamma$ 
		\State $s\gets$ $n\gets$ $0$
		\For{nodes $i\in N$} 
			\If {$k_i\geq k_{min}$}
				\State $s\gets$ $s+\ln k_i/(k_{min}-0.5)$
				\State $n\gets$ $n+1$
			\EndIf
		\EndFor
		\State \Return $1+ns^{-1}$
	\end{algorithmic} \vspace{8pt}
\end{algorithm}

\lipsum[3-4]

}

% \acknow{The authors would like to\dots}

% \showacknow{}

\bibliography{bibliography}

\end{document}
