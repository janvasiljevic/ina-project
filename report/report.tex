\documentclass[9pt,twocolumn,twoside]{pnas-report}

\usepackage{todonotes}

\setuptodonotes{inline,size=\small,color=blue!40}

\templatetype{pnasresearcharticle}

\usepackage{lipsum}


% set figures directory to be ./figures
\graphicspath{{./figures/}}

\title{\textcolor{red}{Draft} Security Analysis of Open-Source Software Package Ecosystems}

\author[a]{Žiga Trček}
\author[a]{Matej Urbas}
\author[a]{Jan Vasiljević}

\affil[a]{University of Ljubljana, Faculty of Computer and Information Science, Ve\v{c}na pot 113, SI-1000 Ljubljana, Slovenia}

\leadauthor{Jan Vasiljević}

\authordeclaration{All authors contributed equally to this work.}
\correspondingauthor{\textsuperscript{1}To whom correspondence should be addressed. E-mail: fine.author@email.com.}

\begin{abstract}
The use of open-source packages and libraries significantly accelerates software development but simultaneously introduces numerous security risks.
Motivated by a recent near-compromise of OpenSSH by a malicious actor, this study aims to investigate the security of open-source software by conceptualizing it as a network and examining transitive vulnerabilities.
Our analysis specifically focuses on PyPI, npm, and crates.io, which are the predominant package managers for Python, JavaScript, and Rust, respectively.
Through this exploration, we seek to uncover potential security weaknesses within these ecosystems and propose methods to enhance their security posture.

\todo{Ni se koncano: Cakam da mamo dejansk neki narejeno. Bolj placeholder.}
\end{abstract}
	
\dates{The manuscript was compiled on \today}
\doi{\href{https://ucilnica.fri.uni-lj.si/course/view.php?id=183}{Introduction to Network Analysis} 2023/24}

\begin{document}

\maketitle
\thispagestyle{firststyle}
\ifthenelse{\boolean{shortarticle}}{\ifthenelse{\boolean{singlecolumn}}{\abscontentformatted}{\abscontent}}{}

\dropcap{T}he date is March 28, 2024. A principal software engineer at Microsoft notices an unusual delay in his login attempts, timing at approximately 500ms—significantly longer than usual by his standards. This prompts an investigation into higher than normal CPU usage, during which he observes anomalous behavior in the SSH daemon process. This unexpected discovery leads to the identification of a vulnerability in the XZ Utils library utilized by OpenSSH. The breach, orchestrated by a malicious actor through a combination of social engineering, code injection, and obfuscation, was set into motion over several years. Interestingly, the core code of OpenSSH remained untouched; instead, a transitive dependency was exploited. Had this vulnerability remained undetected, it could have potentially compromised a vast number of servers relying on OpenSSH, embedding a remote code execution backdoor.

In a manner similar to the exploitation of the XZ Utils library, other open-source software packages and libraries are also vulnerable to malicious attacks.
The development of software inherently involves placing trust in the authors of utilized libraries, who are often individual hobbyists or small teams with limited resources.
The potential for significant damage is large if a malicious actor targets a lesser-maintained package that, while small, is widely used—either directly or as a transitive dependency in other software projects.
The incident involving the left-pad package, which was removed from npm and consequently led to the failure of numerous dependent packages, serves as a reminder of the inherent fragility within the open-source ecosystem.

Int this project we look at three different pacakge ecosystems: PyPI, npm, and crates.io, which are package repositories for Python, JavaScript, and Rust, respectively. We aim to investigate the security of these ecosystems by conceptualizing them as networks.... \textcolor{red}{Zej ne bom pisal dokler ne dejansko necesa nardimo.}

\todo{Problem definition, motivation, background, contributions etc. + Mandatory informative illustration highlighting main contributions}


\section*{Related work}

\todo{Cca 10 referenc.}


\nocite{Kle00,Bou05,EB07,New08,For10,New12,FH16,PLC17,PDL18,Pei20}

\section*{Results}

 {\bf Main results supported by math, plots, tables, diagrams etc.}
\lipsum[1]

\begin{table}[h]\centering%
	\caption{Table describing data or methods.}
	\begin{tabular}{lccccc}\toprule
		             & $n$        & $m$           & $\langle k\rangle$ & $\langle C\rangle$ & $\langle d\rangle$ \\\midrule
		Fine network & $438\,920$ & $9\,742\,733$ & $44.4$             & $0.37$             & $6.19$             \\
		Random graph & $438\,920$ & $9\,781\,609$ & $44.6$             & $0.00$             & $4.92$             \\\bottomrule
	\end{tabular}
	\label{tbl:example}
\end{table}

\lipsum[2-3]

\begin{figure}[t]\centering%
	\includegraphics[width=0.49\linewidth]{example1}
	\includegraphics[width=0.49\linewidth]{example2}
	\caption{Figure showing interesting examples.~\cite{Sub18a}}
\end{figure}

\lipsum[4-6]

\begin{figure}[b]\centering%
	\includegraphics[width=\linewidth]{distributions}
	\caption{Figure showing relevant results.~\cite{Sub18a}}
\end{figure}

\begin{figure}[t]\centering%
	\includegraphics[width=0.45\linewidth]{results1}\hskip12pt
	\includegraphics[width=0.45\linewidth]{results2}
	\caption{Another figure with results.~\cite{Sub18a}}
	\label{fig:example}
\end{figure}

\section*{Discussion}

 {\bf Summary of results, main contributions, final conclusions, future work etc.}
\lipsum[1-3]

{\small

	\section*{Methods}

	 {\bf Data, methods, algorithms etc.}
	\lipsum[1]

	\begin{equation}
		\phi_v = \Pr(X_{st}(v) = 1) = \Pr(X_{sv} = 1)\Pr(X_{vt} = 1)
		\label{eq:example}
	\end{equation}

	\lipsum[2]

	\begin{algorithm}[H]
		\begin{algorithmic}[1]
			\Require graph $G$, cutoff $k_{min}$
			\Ensure power-law $\gamma$
			\State $s\gets$ $n\gets$ $0$
			\For{nodes $i\in N$}
			\If {$k_i\geq k_{min}$}
			\State $s\gets$ $s+\ln k_i/(k_{min}-0.5)$
			\State $n\gets$ $n+1$
			\EndIf
			\EndFor
			\State \Return $1+ns^{-1}$
		\end{algorithmic} \vspace{8pt}
	\end{algorithm}

	\lipsum[3-4]

}

% \acknow{The authors would like to\dots}

% \showacknow{}

\bibliography{bibliography}

\end{document}